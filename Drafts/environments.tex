%% Mathematical Operators and Characters
\newcommand{\N}{\mathbb{N}}
\newcommand{\Z}{\mathbb{Z}}
\newcommand{\R}{\mathbb{R}}
\newcommand{\C}{\mathbb{C}}
\newcommand{\Sph}{\mathbb{S}}
\newcommand{\GL}{\text{GL}}
\newcommand{\Id}{\text{Id}}
\newcommand{\indep}{\perp \!\!\! \perp}
\DeclareMathOperator{\Var}{Var}
\DeclareMathOperator{\Cov}{Cov}
\DeclareMathOperator{\Ex}{\mathbb{E}}
\DeclareMathOperator{\Prb}{\mathbb{P}}
%\DeclareMathOperator{\H}{\mathbf{H}}
\DeclareMathOperator*{\argmax}{arg\,max}
\DeclareMathOperator*{\argmin}{arg\,min}
\DeclareMathOperator{\Div}{div}
\DeclareMathOperator{\supp}{supp}
\DeclareMathOperator{\bd}{bd}
\DeclareMathOperator{\cl}{cl}
\DeclareMathOperator{\Bern}{Bern}
\DeclareMathOperator{\logit}{logit}
\DeclareMathOperator{\ini}{in_{\prec}}


\declaretheorem[name=Theorem, numberwithin=subsection]{theorem}

\declaretheorem[name=Lemma, sibling=theorem]{lemma}

\declaretheorem[name=Observation, sibling=theorem]{obs}

\declaretheorem[name=Conjecture, sibling=theorem]{conj}

\declaretheorem[name=Example, sibling=theorem]{example}

\declaretheorem[name=Definition, sibling=theorem]{definition}

\declaretheorem[name=Proposition, sibling=theorem]{prop}

\declaretheorem[name=Corollary, sibling=theorem]{cor}

\declaretheorem[name=Remark, sibling=theorem]{remark}

\declaretheorem[name=Question, sibling=theorem]{question}

\numberwithin{equation}{subsection}
