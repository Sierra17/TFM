\documentclass[12pt]{article}

\usepackage[utf8]{inputenc}
\usepackage[T1]{fontenc}
\usepackage{amsthm, amsmath, amssymb, amsfonts, mathtools, mathrsfs}
\usepackage{bbm}
\usepackage{dsfont}
\usepackage{thmtools,thm-restate}
\usepackage{graphicx, calc}
\usepackage{minted}

\usepackage[left=2cm,right=2cm,top=2.5cm,bottom=3cm]{geometry}
\usepackage{tikz}
\usepackage{tikz-cd}
\usepackage{pgfplots}
\pgfplotsset{compat=1.15}
\usepackage[shortlabels]{enumitem}
\usepackage{empheq}
\usepackage{comment}
\usepackage[style=english]{csquotes}
\usepackage{xspace}
\usepackage{siunitx}
\usepackage{fancyhdr}
\usepackage{caption}
\usepackage{placeins} % FloatBarrier
\usepackage{physics}
\usepackage{ragged2e}
\usepackage{multicol}
\usepackage{lipsum}
\usepackage{accents}
\usepackage{algorithm}
\usepackage{algpseudocode}
\usepackage{cancel}
\renewcommand\labelenumi{(\theenumi)} %round parentheses

\usepackage[sorting=none,style=authoryear]{biblatex}
\addbibresource{references.bib}

% Links
\usepackage{hyperref}
\hypersetup{
    colorlinks = true,
    filecolor = magenta,
    linkcolor = blue
}

% Colours
\usepackage{xcolor}
\definecolor{RawSienna}{HTML}{882D17}
\definecolor{Salmon}{HTML}{FA8072}

\usetikzlibrary{arrows}

\tikzset{
  treenode/.style = {align=center, inner sep=0pt, text centered,
    font=\sffamily},
  arn_n/.style = {treenode, circle, white, font=\sffamily\bfseries, draw=black,
    fill=black, text width=1.5em},% arbre rouge noir, noeud noir
  arn_r/.style = {treenode, circle, black, draw=black, 
    text width=1.5em, very thick},% arbre rouge noir, noeud rouge
  arn_x/.style = {treenode, rectangle, draw=black,
    minimum width=0.5em, minimum height=0.5em},% arbre rouge noir, nil
  arn_w/.style = {treenode, circle, draw = white, minimum width=0.5em, minimum height=0.5em}% empty node
}

%% Mathematical Operators and Characters
\newcommand{\N}{\mathbb{N}}
\newcommand{\Z}{\mathbb{Z}}
\newcommand{\R}{\mathbb{R}}
\newcommand{\C}{\mathbb{C}}
\newcommand{\Sph}{\mathbb{S}}
\newcommand{\GL}{\text{GL}}
\newcommand{\Id}{\text{Id}}
\newcommand{\indep}{\perp \!\!\! \perp}
\DeclareMathOperator{\Var}{Var}
\DeclareMathOperator{\Cov}{Cov}
\DeclareMathOperator{\Ex}{\mathbb{E}}
\DeclareMathOperator{\Prb}{\mathbb{P}}
%\DeclareMathOperator{\H}{\mathbf{H}}
\DeclareMathOperator*{\argmax}{arg\,max}
\DeclareMathOperator*{\argmin}{arg\,min}
\DeclareMathOperator{\Div}{div}
\DeclareMathOperator{\supp}{supp}
\DeclareMathOperator{\bd}{bd}
\DeclareMathOperator{\cl}{cl}
\DeclareMathOperator{\Bern}{Bern}
\DeclareMathOperator{\logit}{logit}
\DeclareMathOperator{\ini}{in_{\prec}}


\declaretheorem[name=Theorem, numberwithin=subsection]{theorem}

\declaretheorem[name=Lemma, sibling=theorem]{lemma}

\declaretheorem[name=Observation, sibling=theorem]{obs}

\declaretheorem[name=Conjecture, sibling=theorem]{conj}

\declaretheorem[name=Example, sibling=theorem]{example}

\declaretheorem[name=Definition, sibling=theorem]{definition}

\declaretheorem[name=Proposition, sibling=theorem]{prop}

\declaretheorem[name=Corollary, sibling=theorem]{cor}

\declaretheorem[name=Remark, sibling=theorem]{remark}

\declaretheorem[name=Question, sibling=theorem]{question}

\numberwithin{equation}{subsection}


\title{TFM Notes}
\author{Luis Sierra Muntané}
\date{2023/2024}

\begin{document}

\tableofcontents
\newpage

\section{Introduction (November 7th, 14th)}

Meeting Ariadna, lierature review and first discussions about the project's scope. Talks about WFAs, Hankel matrices, Sequence distributions and Matrix completion schemes.

\subsection{WFAs}

\begin{definition}
    A Weighted Finite Automaton (WFA) is a tuple $\left(Q, \Sigma,\Delta, i, f\right)$ where
    \begin{itemize}
        \item $Q$ is a finite set of states.
        \item $\Sigma$ is a finite set of symbols (alphabet). The set of all finite strings over this alphabet is usually denoted $\Sigma^\star$. Then, $\sigma\in\Sigma^\star$ is an arbitrary string while $\lambda$ will denote the empty string.
        \item $\Delta: Q\times \Sigma \times \R \rightarrow Q$ is a transition function such that $\Delta(q, \sigma, w) = q'$ and so moves between states.
        \item $i$ is the initial state, given as a function or a state in $Q$.
        \item $f$ is the final state, and can again be given as a function or an element of $Q$.
    \end{itemize}
\end{definition}

\noindent From these WFAs, we can construct what are known as \emph{Hankel matrices}, with which we aim to encode functions over our strings. In fact, let $f:\Sigma^{\star}\rightarrow \R$ denote a function over our set of finite strings (e.g. the density or probability mass), then the Hankel matrix $\mathbf{H}_f := \mathbf{H}(f)$ is the bi-infinite matrix with real entries such that $\mathbf{H}_f(u,v) = f(uv)$ for any $u,v\in\Sigma^{\star}$, where by $uv$ we mean the concatenation of strings $u$ and $v$. Note then that rows are indexed by prefixes and columns by suffixes.

\begin{obs}
    A Hankel matrix has many redundancies in terms of information representation, starting from the fact that $f(\mathbf{x}) = \mathbf{H}_f(\mathbf{x},\lambda) = \mathbf{H}_f(\lambda, \mathbf{x})$, but also, if we decompose the string into its characters $\mathbf{x} = x_1x_2\cdots x_n$ then we may say that $f(\mathbf{x}) = \mathbf{H}_f(x_1\cdots x_i,x_{i+1}\cdots x_n)$ for any $i\in\left\{1,\dots,\left|x\right|-1\right\}$, so fact the value $f(\mathbf{x})$ appears in the matrix $\left|x\right| + 1$ times.
\end{obs}

In practice, we will need to consider finite submatrices of $\mathbf{H}$, for which we can define a \emph{closed sub-basis} $\left(\mathcal{P},\mathcal{S}\right)$

\subsection{Learning WFAs}

The general framework for learning WFAs from Hankel matrices is taken from \cite{ballewfalearning}. There, the authors propose, given a sample $Z = \left(z_1,\dots,z_m\right)$ with samples $z=(x,y)$ where $x\in\mathcal{PS}$ and $y\in\R$ is the observed relative frequency, the regularised procedure to learn the Hankel operator $h$ is that of solving the problem:

\begin{equation*}
    F_Z(h) = \tau N(h) + \Hat{R}_{\Tilde{Z}}(h) = \tau\lVert h\rVert_{p}^2 + \frac{1}{m}\sum_{(x,y)\in Z}{\ell\left(h(x),y\right)}
\end{equation*}

\noindent so that when we take $\ell$ to be a convex function, then the optimisation problem will be convex and will produce

\begin{equation*}
    h_Z = \argmin_{h}{F_Z(h)}    
\end{equation*}

We will see later that an alternating scheme will make more sense than simply solving this convex optimisation problem.

\subsection{Matrix Completion Basics}

Let $A\in\mathcal{M}\left(m\times n, r\right)$ denote an $m\times n$ matrix with rank at most $r$. This space $\mathcal{M}\left(m\times n, r\right)$ is usually referred to as the \emph{determinantal variety} over a given field, for us mainly $\R$ or $\C$, and satisfies a number of definitional properties.

\begin{prop} The determinantal variety satisfies the following properties
    \begin{enumerate}
        \item[$(i)$] $\mathcal{M}\left(m\times n, r\right)$ is the image of the map $\Gamma:U,V\mapsto UV^{\top}$
        \item[$(ii)$] The variety $\mathcal{M}(m\times n;r)$ has dimension
        \begin{equation*}
            d_r(m,n) = \begin{cases}
                r(m+n-r) , \text{ if } m\ge r \text{ and } n\ge r, \quad \text{(think Grassmanian)}\\
                mn, \text{ otherwise}.
            \end{cases}
        \end{equation*}
        \item[$(iii)$] Every $(r+1)\times(r+1)$ minor is equal to zero.
    \end{enumerate}
\end{prop}

\cite{MekaJainDhillon}

\begin{remark}
    The norm of a matrix may be defined in a number of different ways. A common general norm is the \emph{schatten $p$-norm}, defined for a matrix $\mathbf{M}$ by 
    \begin{equation*}
        \lVert \mathbf{M}\rVert_p = \left(\sum_{i=1}^{r}{\sigma_i(\mathbf{M})^p}\right)^{1/p},
    \end{equation*}
    where $\sigma_i(\mathbf{M})$ is its $i$-th largest singular value, so in particular, when $p=2$, this corresponds to the Frobenius norm.
\end{remark}

To decompose, or factorise a matrix of low rank $A\in\mathcal{M}(m\times n, r;\R)$ into two matrices $U \in \mathcal{M}(m\times r, r;\R)$ and $ V\in \mathcal{M}(n\times r, r;\R)$ where we know in advance, or at least we want to impose the rank of the matrix $A$ as being $r$, we can solve the minimisation problem

\begin{equation}
    \min_{U\in\R^{m\times r},V\in\R^{n\times r}}{\lVert A-UV^T\rVert_F},
\end{equation}
where the rank constraint is already satisfied by the domain of choice for $U,V$, but for a rank $k<r$ solution, an extra constraint would be required. This problem is not convex, but by breaking it down into a sequence of successive convex problems we can design a procedure to recover the decomposition.

\begin{algorithm}
    \caption{Alternating Minimisation for Matrix Factorisation \label{alternating_optimisation}}
    \begin{algorithmic}[1]
        
        \State Initialise matrix $V$
        \While{$\quad \lVert A - UV^T \rVert > \varepsilon\:$} :
            \State $U^{i+1} = \argmin_{U}{\lVert A - UV^T\rVert}$
            \State $V^{i+1} = \argmin_{V}{\lVert A - UV^T\rVert}$
            \State $i \leftarrow i + 1$
        \EndWhile
    \end{algorithmic}
\end{algorithm}

\section{November 21st}

\subsection{Hankel Matrix Completion}

For the specific task of Hankel matrix completion, we shall assume a basis $\mathcal{B}= \left(\mathcal{P},\mathcal{S}\right)$ and a sample $Z$


\section{February 6th}

\subsection{Matrix Completion as Error Correction}

Consider a restriction sampling operator $(S_T\mathbf{x}):\R^{2^r}\rightarrow \R^T$ such that $(S_T\mathbf{x})(j) = x_j,\:j\in T$ where $T$ is a random subset of indices in $\left\{1,\dots, 2^r\right\}$

For an $m\times n$ matrix $A$ and a target rank $k$, take $r = \lceil \log_2{n} \rceil$ as the dimension of the code and $\ell > k$ as the length of the code, then we can form a sampling matrix as follows:

\begin{equation}
    \Omega = \sqrt{\frac{2^r}{\ell}}DS\Phi
\end{equation}

\section{February 20th}

\subsection{Matrix completion on graphs}



\newpage
\printbibliography

\end{document}
